\documentclass{article}
\usepackage[utf8]{inputenc}
\usepackage[english]{babel}

\usepackage{multicol}
\usepackage{algorithm}

\title{Structural Inspection Planning for a Mobile Manipulator Robot}
\author{Andrew Washburn, Prateek Arora, Nikhil Khedekar}

\begin{document}
\maketitle

\begin{multicols}{2}

\begin{abstract}
    This report details a 3D structural inspection planning algorithm for an autonomous mobile manipulator robot. The algorithm inputs apriori a polyhedron model of an object to inspect, and viewable constraints, and outputs a near optimal set of viewpoints that result in complete coverage, subject to the mobile robot's constraints. 
\end{abstract}

\section{Introduction}

Structural inspection provides detailed information about a large object, such as a ship hulls or cell towers.

For an autonomous robotic inspector, the goal is to view. These inspections are performed by holonomic robots. In water, a submersible robot can position itself in six degrees of freedom, and in the air, a quadrotor UAV can likewise maneuver without constraint. Structures not in water, or easily accessible from air, such as tight corners, low ceilings, or cluttered environments, require another type of robotic platform. 

A mobile manipulator (a wheeled robot with an attached mechanical arm) can view constrained poses such as tight corners and underneath obstacles. A structure with this geometry is a supported pressure vessel elevated above the ground that has underneath surfaces and complicated faces. 

Inspection algorithms for complete coverage begin with analyzing the structure to inspect \cite{latombe}

Talk about inspection planning, why it's useful, how it's used, and related work. Why is a mobile robot different than other styles?

\section{Viewpoint Selection Algorithm}

This section details the viewpoint selection algorithm.

\begin{algorithm}[H]
    \SetAlgoLined
    \KwData{Polygon, constraints}
    \KwData{Viewpoint Pose set, Base Pose set}
    initialization\;
    \While(){There are unseen points}{
        Compute visible region\;
        Build a lot of stuff\;
        Build more stuff\;
    }
\end{algorithm}

\end{multicols}

\bibliography{references}
\bibliographystyle{plain}

\end{document}